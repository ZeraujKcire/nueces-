
\chapter{METODOLOGÍA CONDSIDERACIONES Y SUPUESTOS.} %(((

Para el desarrollo del presente avalúo, se llevaron a cabo el Enfoque de Costos
a partir de datos obtenidos de proveedores y, en caso de equipos para los cuales
se encontraron comparables, se llevó a cabo el Enfoque de Mercado, utilizando un
modelo de regresión lineal múltiple.

\begin{tabular}{|cp{7cm}|m{2cm}|m{2cm}|}
	\hline 
  	& Nombre & Enfoque de Mercado & Enfoque de Costos \\ \hline 
	1. & Tanque para aire comprimido & Sí & \\ \hline
	2. & Secador de aire caliente & & Sí \\ \hline
	3. & Polipasto eléctrico & Sí & \\ \hline
	4. & Máquina embolsadora & & Sí \\ \hline
	5. & Máquina seleccionadora de granos y semillas por color & & Sí \\ \hline
	6. & Máquina seleccionadora de granos y semillas por color con plataforma & & Sí \\ \hline
	7. & Bombo batidor/mezclador para cofitería & Sí & \\ \hline
	8. & Compresor horizontal oil-free & Sí & \\ \hline
	9. & Máquina de aspersión de poliuretano & & Sí \\ \hline
	10. & Secador de aire tipo refrigerado & Sí & \\ \hline
	11. & Compresor tipo tornillo & Sí & \\ \hline
	12. & Línea embasadora de 24 cabezales & & Sí \\ \hline
	13. & Línea embasadora de 14 cabezales & & Sí \\ \hline
\end{tabular}

\section{ENFOQUE DE COSTOS.} % (((
Para la estimación de costos se  obtuvo información  respecto a cotizaciones de equipos nuevos con las mismas características halladas en páginas especializadas con activos del mismo tipo. Una vez obtenida  dicha información se consideraron las especificaciones proporcionadas por el solicitante.
Posteriormente se aplicaron los factores de la vida útil, fletes y aranceles para obtener el VNR.
% )))

\section{ENFOQUE DE MERCADO.} % (((

Para los activos enlos que se encontró mercado, se realizó su estimación en un modelo de regresión lineal múltiple considerando 
las características específicas de cada activo. \\ 
El análisis estadístico se realiza en todas las pruebas con el mismo nivel de 
significancia: \(1- \alpha\). \\[2mm]
La comprobación de que el modelo de regresión lineal múltiple es significativo
requiere de las siguientes pruebas estadísticas:
\begin{enumerate}
	\item \textbf{Matriz de Dispersión:} Se observan gráficamente las tendencias 
		entre los datos y las correlaciones lineales individuales entre las 
		variables utilizadas.
		\newpage
	\item \textbf{Supuestos del Modelo de Regresión.}
		\begin{enumerate}[i.]
			\item \textbf{Homocedasticidad:} Se realiza el test Breusch-Pagan para 
				comprobar si la varianza de los residuales (\(Y_i\) estimada
				\( - Y_i\) real) es constante.
			\item \textbf{Independencia:} Se realiza el test de Durbin-Watson para
				verificar que las variables tomadas son independientes estadísticamente.
			\item \textbf{Normalidad:} Se realiza el test de Shapiro-Wilk
				para comprobar que los residuos posean una distribución normal.
				(\(Y_i\) estimada \(- Y_i\) real \(\sim N(0, \sigma ^2)\), donde 
				\(\sigma ^ 2\) es la varianza de los datos \(Y_i\)).
		\end{enumerate}
	\item \textbf{Tabla ANOVA:} Se calcula la tabla ANOVA para la obtención del 
		estadístico \(F\) que se usa en la prueba de significancia.
	\item \textbf{Prueba de Significancia del Modelo:} Se usa el estadístico \(F\) 
		para obtener el percentil (valor \(p\))
		y observar si cae dentro de la región de rechazo.
		En caso de que el valor \(p\) caiga en la región de rechazo \((0, \alpha)\)
		se considera que el modelo es significativo.
\end{enumerate}
Se escoge un nivel de confianza del \(90\%\). Entonces \(1- \alpha =0.9\), y 
\(\alpha = 0.1\).
% )))

\espacio

% )))
