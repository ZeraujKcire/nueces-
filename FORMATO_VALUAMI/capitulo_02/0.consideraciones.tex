\chapter{CONSIDERACIONES.} % (((

\section{ENTORNO DE LOS BIENES VALUADOS.} % (((
\begin{tabular}{r @{\bf : \hspace{5mm}} p{0.4 \linewidth}}
	\bf Producto que se obtiene & Clasificación y embolsado de nuez. \\ 
	\bf Giro Industrial         & \\ 
	\bf Subsector               & 311 Industria alimentaria. \\ 
	\bf Rama económica          & 3119 Otras industrias alimentarias. \\ 
	\bf Clase de actividad      & El equipo está fuera de uso en bodega, por el 
	que no fue identificado algún impacto en el medio ambiente.
\end{tabular}
% )))

\section{CONDICIÓN GENERAL DE LOS BIENES.} % (((
\begin{tabular}{p{0.3 \linewidth} p{0.65 \linewidth}}
	\bf Tiempo de Operación:      & Por información proporcionada al momento de la visita los equipos estaban en posición final, sin instalar ni operar. \\
	\bf Procedencia:              & China.  \\
	\bf Operación:                & Desinstalados y fuera de operación \\  
	\bf Capacidad:                & La capacidad de producción del equipo está descrita individualmente en el apartado de maquinaria y equipo. \\
	\bf Condición General:        & Por información proporcionada al momento de la visita, los equipos estaban en posción final, sin instalar ni operar. \\
	\bf Tipo de Mantenimiento:    & Por información proporcionada al momento de la visita, los equipos estaban en posción final, sin instalar ni operar. \\
	\bf Frecuencia:               & Por información proporcionada al momento de la visita, los equipos estaban en posción final, sin instalar ni operar. \\
	\bf Calificación:             & Por información proporcionada al momento de la visita, los equipos estaban en posción final, sin instalar ni operar. \\
	\bf Datos de Mantenimiento:   & Por información proporcionada al momento de la visita, los equipos estaban en posción final, sin instalar ni operar. \\
	\bf Inovaciones Tecnológicas: & No existen innovaciones tecnológicas, actualmente se fabrican equipos con tecnología similar. \\
	\bf Tipo de Mercado:          & Los bienes pueden ser comercializados en un mercado abierto nacional como internacional dentro del sector al que pertenecen dichos equipos. 
	Se hace notar que para este tipo de equipos de manufactura China, en la investigación realizada no se detectó mercado de segunda mano. 
	El mercado de segunda mano que fue detectado tiene manufactura europea (Alemana) y este supera el Valor de Reposición Nuevo de los equipos sujetos. 
	Por tanto, no fue aplicado.
\end{tabular}
% )))

% )))
